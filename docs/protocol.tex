\documentclass[11pt,a4paper]{article}
\usepackage[utf8]{inputenc}
\usepackage[english]{babel}
\usepackage{amsmath}
\usepackage{amsfonts}
\usepackage{amssymb}

\usepackage{listings}
\usepackage{xcolor}

\colorlet{punct}{red!60!black}
\definecolor{background}{RGB}{252,252,252}
\definecolor{delim}{RGB}{20,105,176}
\colorlet{numb}{magenta!60!black}

\lstdefinelanguage{json}{
    basicstyle=\normalfont\ttfamily,
    numbers=left,
    numberstyle=\scriptsize,
    stepnumber=1,
    numbersep=8pt,
    showstringspaces=false,
    breaklines=true,
    frame=lines,
    backgroundcolor=\color{background},
    literate=
     *{0}{{{\color{numb}0}}}{1}
      {1}{{{\color{numb}1}}}{1}
      {2}{{{\color{numb}2}}}{1}
      {3}{{{\color{numb}3}}}{1}
      {4}{{{\color{numb}4}}}{1}
      {5}{{{\color{numb}5}}}{1}
      {6}{{{\color{numb}6}}}{1}
      {7}{{{\color{numb}7}}}{1}
      {8}{{{\color{numb}8}}}{1}
      {9}{{{\color{numb}9}}}{1}
      {:}{{{\color{punct}{:}}}}{1}
      {,}{{{\color{punct}{,}}}}{1}
      {\{}{{{\color{delim}{\{}}}}{1}
      {\}}{{{\color{delim}{\}}}}}{1}
      {[}{{{\color{delim}{[}}}}{1}
      {]}{{{\color{delim}{]}}}}{1},
}


\author{marckolak}
\title{Wild Thumper project communication protocol}
\begin{document}
\maketitle

\section{Introduction}
The communication between the Raspberry Pi based controller is performed over WiFi using TCP. The controller offeres a single TCP socket so only one client might be connected.

The packets include a JSON formatted payload. The payload should have the following structure:
\vspace{6pt}
\begin{lstlisting}[language=json,firstnumber=1]
{
	"cmd": str,
  	"payload": "file"
}
\end{lstlisting}
where \texttt{cmd} is the command name and \texttt{payload} includes command parameters e.g. direction of motion, number of scans to be taken.

\section{Commands}
The following section includes the description of the commands, which are the controller is able to interpret and execute.

\subsection{Movement commands - \texttt{move}}
The movement commands have the following structure:
\vspace{6pt}
\begin{lstlisting}[language=json,firstnumber=1]
{
	"cmd": "move",
  	"payload": {
		"dir": str,
		"speed": number,
		"time": number,
		"args": object
  	}
}
\end{lstlisting}


\subsection{SLAM commands}
\subsubsection{Start/Stop scanning}
In order to start or stop LiDAR operation the client must send a message with scan command with \texttt{"scan"} command.
\vspace{6pt}
\begin{lstlisting}[language=json,firstnumber=1]
{
	"cmd": "scan",
  	"payload": {
		"action": "start"/"stop",
		"speed": number,
		"rate": number
  	}
}
\end{lstlisting}

\section{Responses}

\subsection{Scan measurement data}
When starting the LiDAR the client might ask to send data. After each of the scans the data will be sent to the client in the following format:

\vspace{6pt}
\begin{lstlisting}[language=json,firstnumber=1]
{
	"cmd": "scan",
  	"payload": {
  		"timestamp": number,
		"x": array,
		"y": array
  	}
}
\end{lstlisting}

\end{document}